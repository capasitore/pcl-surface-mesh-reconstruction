\newpage
\thispagestyle{empty}

\section*{Prilozi} % (fold)
\addcontentsline{toc}{section}{Prilozi}
\label{sec:Prilozi}

\subsection*{Prevođenje i instalacija RGBDSlam programa} % (fold)
\label{sub:RGBDSlam prilog}

Prevođenje i instalacije programa rađena je na Ubuntu 12.04 GNU/Linux
distribuciji pomoću skripte prikazane u ispisu~\ref{skripta} Upute za
ostale distribucije se nalaze na \url{http://wiki.ros.org/rgbdslam}.
Skripta automatizira dodavanje ROS repozitorija i poziva instalaciju
svih potrebinh paketa. Zatim postvlja ROS varijable okruženja potrebne za
pokretanje RGBDSlam programa. Tada se klonira izvorni kod programa i
pozivaju naredbe za instaliranje i prevođenje programa.

\begin{lstlisting}[language=bash,label=skripta, keywords={sudo, wget, 
    echo, apt-get, rosdep, roscd, rosmake}, caption={Ispis shell skripte 
    za instalaciju rgbdslam programa}]
# Better pick a mirror close to you. 
# See http://ros.org/wiki/ROS/Installation/UbuntuMirrors
sudo sh -c '. /etc/lsb-release && echo "deb http://packages.ros.org/ros/ubuntu $DISTRIB_CODENAME main" > /etc/apt/sources.list.d/ros-latest.list' 

wget http://packages.ros.org/ros.key -O - | sudo apt-key add -

sudo aptitude update

# This will draw gigabytes from the network:
sudo apt-get install ros-fuerte-perception-pcl ros-fuerte-vision-opencv ros-fuerte-octomap-mapping python-rosdep 
sudo apt-get install ros-fuerte-openni-launch

echo 'source /opt/ros/fuerte/setup.bash' >> ~/.bashrc
echo 'export ROS_PACKAGE_PATH=~/ros:$ROS_PACKAGE_PATH' >> ~/.bashrc
. ~/.bashrc

svn co http://alufr-ros-pkg.googlecode.com/svn/trunk/rgbdslam_freiburg ~/ros/rgbdslam_freiburg

sudo rosdep init
rosdep update
rosdep install rgbdslam_freiburg
roscd rgbdslam

# This will take a while:
rosmake rgbdslam_freiburg
\end{lstlisting}


% subsection RGBDSlam prilog (end)

% section Prilozi (end)
