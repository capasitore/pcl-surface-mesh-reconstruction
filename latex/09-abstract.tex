\newpage
\thispagestyle{empty}

\section*{Sažetak} % (fold)
\addcontentsline{toc}{section}{Sažetak}
\label{sec:Sažetak}

% Sažetak s ključnim riječima na hrvatskom jeziku. Naslov i sažetak s
% ključnim riječima na engleskom jeziku.  Opisati glavni problem,
% naznačiti smjernice kako je rješavan te naznačiti postignute rezultate
% diplomskog rada. Sažetke dati na 10 do 15 redaka (do pola stranice).

Izgradanja 3D modela scene pomoću 3D kamere je metoda koja upotrebljava
nekoliko komplementarnih tehnologija. U radu je razvijen i predstavljen
program za izgradnju mreže trokuta iz snimljenog oblaka točaka te je
ispitana kvalteta i funkcionalnost razvijene metode izgradnjom nekoliko
3D modela objekata i scena.  Snimanje scena se izvršava upotrebom
RGBDSlam programa i Microsoft Kinect kamere. RGBDSlam program kontrolira
uzimanje slika s kamere i od njih sastavlja oblak točaka koji prikazuje
3D scenu upotrebom tehnike istovremene lokalizacije i mapiranja - SLAM.
Program je baziran na ROS programskom okviru i OpenCV biblioteci. Takav
oblak točaka se koristi za izgradnju 3D mreže trokuta razvijenim
programom mesh-reconstruction. mesh-reconstruction se oslanja na PCL
biblioteku u kojoj je implementiran Poisson algoritam za izgradnju mreže
trokuta. Program ima grafičko sučelje razvijeno pomoću Qt okvira koje
omogućava reduciranje i uklanjanje odudarajućih vrijednosti iz
snimljenog oblaka, podešavanje Poisson parametara te izgradnju i prikaz
mreže trokuta.  \\

\noindent\textbf{Ključne riječi:} Kinect, RGBDSlam, ROS, OpenCV, SLAM, PCL,
Poisson, Qt
% section Sažetak (end)

\section*{Abstract} % (fold)
\label{sec:Abstract}
\textbf{Title:} 3D model reconstruction with 3D camera \\

3D model reconstruction with 3D camera is a process that uses a few
complementary technologies. This masters thesis presents developed
application for triangle mesh reconstruction from acquired point cloud.
Thesis also examines quality and functionality of developed method by
reconstructing few 3D models of objects and scenes. Scene recording is
done by using RGBDSlam application with Microsoft Kinect camera.
RGBDSlam controls picture capturing and from them it assemblies point
cloud which presents 3D scene using SLAM technique. Program is based on
ROS framework and OpenCV library. Acquired point cloud is used for
3D mesh reconstruction with developed program mesh-reconstruction.
Program is based on PCL library in which there is implementation of
Poisson algorithm for mesh reconstruction. It has graphical user
interface developed in Qt which features functions for down sampling and
removing outliers from captured point cloud, also it offers Poisson
parameters configuration and has functions for reconstruction and
visualisation of constructed mesh of triangles.  \\

\noindent\textbf{Keywords:} Kinect, RGBDSlam, ROS, OpenCV, SLAM, PCL,
Poisson, Qt

% section Abstract (end)
