\documentclass{beamer}
\usepackage[utf8]{inputenc}
\usepackage[T1]{fontenc} 
\usetheme{Luebeck}
\setbeamercovered{transparent}

\title[Izgradnja 3D modela scene pomoću 3D kamere]{Izgradnja 3D modela
scene pomoću 3D kamere}
\author{Marijan Svalina}
\institute{Elektrotehnički fakultet Osijek \\ Diplomski studij
računarstva}
\date{Srpanj, 2014}
\begin{document}

%%% 1. Slide %%%
\begin{frame}
    \titlepage
\end{frame}


%%% 2. Slide %%%
\begin{frame}{Pregled prezentacije}
    \tableofcontents[currentsubsection, pausesections]
\end{frame}

%%% 3. Slide %%%
\section{Uvod} 
\begin{frame}{Uvod i temelji rada}
    \minipage{0.38\linewidth}
        \includegraphics<1->[width=\linewidth]{../figures/kinect2.png}
    \endminipage
    \minipage{0.32\linewidth}
        \includegraphics<1->[width=\linewidth]{../figures/gpl.png}
    \endminipage
    \minipage{0.23\linewidth}%
        \includegraphics<1->[width=\linewidth]{../figures/bsd.png}
    \endminipage
    \begin{itemize}
        \item <2-> 2010. godina, \alert{Kinect} kamera, Xbox.
        \item <3-> 30Hz, slika u boji sinkronizirana s \alert{dubinskom}
            slikom.
        \item <4-> Prirodniji način rješavanja problema računalnog vida.
        \item <5-> \alert{Slobodan softver}.
    \end{itemize}
\end{frame}

%%% 4. Slide %%%
\subsection{Zadatak i opis projekta}
\begin{frame}{Grafički prikaz zadatka projekta}
    \includegraphics<1->[width=\linewidth]{../figures/project-description.jpeg}
    \begin{itemize}
        \item <2-> Razviti program za izgradnju 3D modela u obliku mreže
            trokuta koristeći PCL biblioteku.
        \item <3-> Ispitati funkcionalnost i kvalitetu postupka kao i
        \item <4-> kvalitetu dobivenog 3D modela izgradnjom nekoliko 3D
            modela objekata i scena. 
    \end{itemize}
\end{frame}

%%% 5. Slide %%%
\section{Pregled upotrebljenih tehnologija i algoritama} 
\begin{frame}
    \tableofcontents[currentsection]
\end{frame}

%%% 6. Slide %%%
\subsection{Microsoft Kinect kamera}
\begin{frame}{Kinect 3D kamera}
    \includegraphics<1->[width=\linewidth]{../figures/kinect.png}
    \begin{itemize}
        \item <2-> Slika u boju sinkronizirana s dubinskom slikom.
        \item <3-> VGA rezolucija pri 30Hz.
        \item <4-> Dubinski senzor:
            \begin{itemize}
                \item <4-> Laserski IR projektor i IR kamera.
                \item <5-> Ograničenje dometa na 0.8m - 3.5m.
            \end{itemize}
    \end{itemize}
\end{frame}

%%% 7. Slide %%%
\begin{frame}{Kinect 3D kamera - princip rada dubinskog senzora}
    \begin{itemize}
        \item <1-> Princip struktrirane svjetlosti.
        \item <2-> IR projektor projicira \alert{jedinstven} uzorak
            točkastih mrlja.
        \item <3-> IR kamera hvata reflektirane IR mrlje.
        \item <4-> Računanje dubine se odvija na kameri stereo
            triangulacijom.
    \end{itemize}
\end{frame}

%%% 8. Slide %%%
\subsection{ROS biblioteka i alati}
\begin{frame}{ROS - Operacijski sustav za robote}
    \includegraphics <1->[width=\linewidth]{../figures/ros.png}
    \begin{itemize}
        \item <2-> "Meta" OS.
        \item <3-> Hardverska apstrakcija, upravljački programi,
            biblioteke, alati, komunikacija, paketi...
        \item <4-> RGBDSlam - baziran na ROSu.
    \end{itemize}
\end{frame}

%%% 9. Slide %%%
\subsection{PointCloud biblioteka}
\begin{frame}{Biblioteka funkcija i algoritama za rad s oblakom
    točaka}
    \includegraphics<1->[width=\linewidth]{../figures/pcl.png}
    \begin{itemize}
        \item <2-> \alert{Oblak točaka} - 3D skup točaka dobiven
            spajanjem RGB slike i dubinske slike.
        \item <3-> Uključena u ROS.
        \item <4-> Upotrebljena za razvoj \alert{\texttt{mesh-reconstruction}}
            programa.
    \end{itemize}
\end{frame}

%%% 10. Slide %%%
\subsection{Istovremena lokalizacija i mapiranje}
\begin{frame}{SLAM - istovremena lokalizacija i mapiranje }
    \begin{itemize}
        \item <2-> Izgradnja karte nepoznatne okoline i istovremena
    navigacija upotrebom te karte.
        \item <3-> Cilj procesa je korištenje percepcije okoline za
            pozicioniranje robota/kamere.
        \item <4-> To se postiže detektiranjem orijentira/značajki.
        \item <5-> Jezgra procesa je EKF - odgovoran za ažuriranje
            pozicije na kojoj robot "misli" da se nalazi.
        \item <6-> EKF rekurzivni je algoritam koji se izvršava na nizu
            šumovitih podataka te računa statistički optimalnu
            estimaciju sustava.
    \end{itemize}
\end{frame}

%%% 11. Slide %%%
\subsection{Poisson algoritam za rekonstrukciju površine}
\begin{frame}{Poisson algoritam za rekonstrukciju površine}
    \includegraphics<1->[width=\linewidth]{../figures/poisson-reconstruction.png}
    \begin{itemize}
        \item <2-> Poissonova jednadžba $\Delta \chi \equiv \nabla \cdot
            \nabla\chi = \nabla \cdot \vec{V}.$
        \item <3-> Postoji veza između normala i indikacijske funkcije.
        \item <4-> Gradijent indikacijske funkcije jednak je unutrašnjim
            normalama površine.
    \end{itemize}
    
\end{frame}

%%% 12. Slide %%%
\section{Snimanje i izgradnja 3D modela scene} 
\begin{frame}
    \tableofcontents[currentsection]
\end{frame}

%%% 13. Slide %%%
\subsection{Snimanje scene RGBDSlam programom} 
\begin{frame}{Opis rada RGBDSlam programa}
    \begin{center}
        \includegraphics <1->[scale=0.16]{../figures/rgbdslam.png}
    \end{center}
    \begin{itemize}
        \item <2-> Računanje i sparivanje značajki.
        \item <3-> 3D korespodencije između sličica.
        \item <4-> Estimiranje poza - RANSAC.
        \item <5-> Optimiziranje grafa poza.
    \end{itemize}
\end{frame}

%%% 14. Slide %%%
\begin{frame}[plain]{Prikaz RGBDSlam programa}
    \begin{center}
        \includegraphics <1->[scale=0.30]{../figures/rgbdslamSS.jpeg}
    \end{center}
\end{frame}

%%% 15. Slide %%%
\subsection{Izgradnja 3D modela scene pomoću mreže trokuta} 
\begin{frame}{Pregled razvijenog programa mesh-reconstruction}
    \begin{center}
        \includegraphics <1->[scale=0.28]{../figures/flowchart.pdf}
    \end{center}
\end{frame}

%%% 16. Slide %%%
\begin{frame}[plain]{Grafičko sučelje programa mesh-reconstruction}
    \begin{center}
        \includegraphics
        <1->[scale=0.32]{../figures/mesh-reconstruction-gui.png}
    \end{center}
\end{frame}

%%% 17. Slide %%%
\begin{frame}[plain]{Grafičko sučelje programa mesh-reconstruction}
    \begin{center}
        \includegraphics <1->[scale=0.27]{../figures/gui-2.png}
    \end{center}
\end{frame}

%%% 18. Slide %%%
\section{Rezultati pokusa} 
\subsection{Prikaz izgrađenih modela scena i objekata}
\begin{frame}{Ispitivanje funkcionalnosti snimanjem nekoliko scena}
    \begin{center}
        \includegraphics <1->[width=\linewidth]{../figures/01-all-pcd.png}
    \end{center}
    \begin{itemize}
        \item <2-> demo...
    \end{itemize}
\end{frame}

%%% 19. Slide %%%
\section{Zaključak} 
\begin{frame}{Nedostatci, prednosti, poboljšanja}
    \begin{itemize}
        \item <1-> Ograničenja tehnologija, Kinect, RGBDSlam, Poisson. 
        \item <2-> Skeniranje, 3D printeri, FPS.
        \item <3-> Poisson, boja, rupe, zatvaranje petlje.
    \end{itemize}
\end{frame}

%%% 20. Slide %%%
\section{Zaključak} 
\begin{frame}{Pitanja?}
    \begin{itemize}
        \item <1-> Hvala na pažnji!
    \end{itemize}
\end{frame}

\end{document}
